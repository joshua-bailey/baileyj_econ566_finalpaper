\section{Concluding Remarks} \label{sec:conc}

In this paper, I have shown that modern machine learning techniques, like RFs, can substantially improve the accuracy of inflation forecasting over traditional AR models. Even though the traditional model reasserted itself over the post-Covid period, the broader literature finds that RF models continue to perform well in a variety of settings. \textcite{Kohlscheen2021WhatInflation} for instance, finds that an RF model predicts inflation across 20 advanced countries between 2000 and 2021 and finds that it outperforms the benchmark AR and OLS models over the three-month forecasting horizon. The Lasso model also generated an important result, by showing over a nearly 10-year period that -- as we would expect from theory -- monetary policy and labour market variables substantially explain the path of inflation. 

RF models, due to their ability to handle large datasets with underlying nonlinearities, are a particularly promising area for further study in macroeconomic forecasting. One recent innovation, not explored in this paper due to time limitations, is the `Macro Random Forest' developed by \textcite{GouletCoulombe2024TheForest}. The MRF framework refines this approach by focusing on modelling time-varying economic coefficients ($\beta_t$), rather than directly predicting the output variable ($y_t$). The interpretability of MRF is a significant advantage, especially during shocks. The Generalised Time-Varying Parameters (GTVPs) it produces serve as a flexible tool, accommodating various forms of nonlinearities such as thresholds, structural breaks, or smooth transitions. This adaptability ensures that the model can dynamically adjust its parameters in response to shock-induced changes in economic relationships. Models like the MRF are promising avenues for further research and show the continued potential of machine learning tools to improve inflation forecasting going forward. 

