\pagebreak
\section{Literature Review: Machine Learning and Inflation Forecasting} \label{sec:lit}


Kleinberg et al. (2015) explore the benefits and limitations of using ML approaches for economic forecasting. They advocate for ML as a disciplined, non-parametric approach to predicting economic outcomes. Mullainathan and Spiess (2017) illustrate how regression trees can enhance predictions of house prices, summarising their findings by asserting that "machine learning offers a potent tool to discern, with unprecedented clarity, the messages conveyed by the data." Thus, ML can serve as a valuable adjunct to traditional model-based methods. Chakraborty and Joseph (2017) evaluate the performance of ten econometric and ML models in predicting inflation in the United Kingdom post-Global Financial Crisis, identifying random forests as the superior model in their test samples. Subsequently, Medeiros et al. (2021) assess various ML techniques for forecasting inflation in the United States, also highlighting the superiority of random forests over competing methods. These results echo broader findings by Fernandez-Delgado et al. (2014), who evaluated 179 classifier models across 121 datasets and found random forests to excel as the top performer. Furthermore, Coulombe (2021) makes an initial effort to integrate random forest methodology with a macroeconomic model.